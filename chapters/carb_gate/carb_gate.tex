INTRO: 
	- Different ways of realizing them
	- CZ gate


\chapter{Calibrating and characterizing controlled arbitrary phase gates}
TODO: read iswap, johannes Thesis ch cphase gate.
- summary chapter
- intro two qubit gate: 
	- Role in algorithm; clifford group (?)  --> just subset is enough for computation, implies decomposition of more complex gates. 
	- In NISQ, this can be a problem due to decoherence
- in specific cases, might be useful to have something else than two standard clifford group to make it gate/time efficient. 
- in this work, investigate one case applied to Variation Quantum Algorithms. Other recent example iswap.

\section{Motivation}
- 
Where can it be used ?
Advantages ?
\section{Theoretical description}
TODO: read Dicarlo, 
- restate CZ hamiltonian, show that by tuning amplitude and interaction time, we can change the phase to arbitrary value. 
- Chevron pattern --> Need high pop recovery.

Interaction hamiltonian and unitary
Fig: ?
\section{Calibration}: Note: All with 3 level RO (no, not dynamic phase) --> is that a problem when leakage non negligeable?)
- Discrete calibration for continuous parameters: linear interpolation between calibration points. Discuss time for calibration here ?
- Pulse parametrization: flat top gaussian. good for chevron but potentially accumulates changes on bias-T (not Net Zero). 
 Check up to which depth we can go. + need to introduce IIR and FIR filters. + buffers.
 Three steps required for calibration:
 Chevron + fit : ==> population recovery
 calibrate conditional phase as function of flux pulse amplitude. 
 Calibrate dynamic phase 
 * Chevron:
- see Fig. Vary pulse amplitude and length. Explain scheme.
- fit cosine to each and take first maximum. Explain "time effective chevron". 
- fit model with free parameters for dphidV. NOTE: where to put derivation of model ? Should I talk about the model (probably).

* Conditional phase:
- calibrate the amplitude yielding a specific conditional phase. Ramsey Experiment with C arb in between. Pulse length adapted automatically.  
- Non linear function of amplitude -> different samplings.
- not zero due to Alphazz even if pulse == 0: from alphazz
 
* Dynamic phase:
TODO: search ref for dyn phase calib.  (DiCarlo, Bauer, Heinsoo)
QUESTION: difference dynamic phase and alphazz? alphazz effect on 11 and dynamic phase on individual ? does calibration suffer from alphazz ? I don't think so because other qubit is ont in 1 state.
NOTE: origin of dynamic phase should ideally already be explained by this point.
- brief recap dynamic phase is. Model. Approximation of square pulse.
- has to be calibrated if gate is to be used in algorithm. We add z gates to compensate for it. 
- calibration scheme: pi half + flux + pi half (i.e. ramsey experiment). (with and without flux?)
- Requires calibration for all possible lengths/amplitudes. Rrequires above niquist sampling rate.
See from model which sampling rate is required.
- Discuss Figure. Deviations from model? 

General discussion on calibration: time.
Link to next: investigate performance, leakage and stability required for use in algorithm.
\section{Characterization/ Benchmarking}
First step is to perform process tomography. Show one in fig + other angles in table or plot. 
Note: has to say that remove leakage. Justification: leakage has unpredictable effect on tomography since not in comp subspace. 
Hence 3L RO, RM F level and consider leakage separately. TODO: ref on influence of leakage in process tomo?
Non clifford --> cannot use RB ? 
Cross entropy Benchmarking: why good and why not?

 C- and D-phase errors: change to gate error. Fig: two in 1?
 Leakage over phi: Discuss tendencies and causes. FIR filters. Length of interaction.
 Process tomography
Time drift (combine this with prev?): Stable over time.